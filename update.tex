% TEMPLATE for Usenix papers, specifically to meet requirements of
%  USENIX '05
% originally a template for producing IEEE-format articles using LaTeX.
%   written by Matthew Ward, CS Department, Worcester Polytechnic Institute.
% adapted by David Beazley for his excellent SWIG paper in Proceedings,
%   Tcl 96
% turned into a smartass generic template by De Clarke, with thanks to
%   both the above pioneers
% use at your own risk.  Complaints to /dev/null.
% make it two column with no page numbering, default is 10 point

% Munged by Fred Douglis <douglis@research.att.com> 10/97 to separate
% the .sty file from the LaTeX source template, so that people can
% more easily include the .sty file into an existing document.  Also
% changed to more closely follow the style guidelines as represented
% by the Word sample file. 

% Note that since 2010, USENIX does not require endnotes. If you want
% foot of page notes, don't include the endnotes package in the 
% usepackage command, below.

% This version uses the latex2e styles, not the very ancient 2.09 stuff.
\documentclass[letterpaper,twocolumn,10pt]{article}
\usepackage{usenix,epsfig,endnotes}
\usepackage{url}
\begin{document}

%don't want date printed
\date{}

%make title bold and 14 pt font (Latex default is non-bold, 16 pt)
\title{\Large \bf Mouse and KeyboardTracking Detection \\ Project Update}

%for single author (just remove % characters)
\author{
{\rm Nathan G. Feldman}\\%in case you're wondering, there's already a few other Nathan Feldmans on Google Scholar, so I need to distinguish myself in scholarly work
University of California, San Diego
\and
{\rm Scott Finney}\\
University of California, San Diego
% copy the following lines to add more authors
% \and
% {\rm Name}\\
%Name Institution
} % end author

\maketitle

% Use the following at camera-ready time to suppress page numbers.
% Comment it out when you first submit the paper for review.
\thispagestyle{empty}


%1. What closely related work, if any, did you find and read?
%2. What things did you try on the project since you proposed it?
%3. What findings did #2 lead to?
%4. How did the related work you read and your findings change your proposal, if at all?
%5. What is your plan for the remainder of the quarter?
%6. What, if anything, do you need from me and Keaton to help you succeed

\subsection*{Abstract}
We aim to design and implement a technique for detecting when websites store data about mouse movement and keyboard presses and send it out over the internet, either back to the website's own server or out to the server of a third-party service.  We would like to build a web crawler to visit as many of the top Alexa websites as possible and use this technique on each of them to find out which ones do mouse and keyboard tracking.

We wish to analyze our results to find out as precisely as possible what information each of them is sending, to whom, and for what purpose it is being sent. In doing so, we will find out how much the typical web users' expectations of privacy on the web are being violated.

Time permitting, we may also implement a browser extension to do detect mouse and keyboard tracking at the user's request.

\section{Goals}
The core of this project is to design and implement a technique for detecting mouse and keyboard tracking on the web. The technique should allow us to find out when a website's client-side script is sending information about the user's mouse movement and keyboard presses out over other the internet. Mouse movement, while not a particularly severe security concern, may violate the typical user's expecation of privacy on the web. Keyboard press detection raises the same concern, but may also reveal sensitive information about the user, especially her password, to malicious sites she visited by accident or by coersion. (Of course, any website where the user has to log in can see the user's password for that site, which is most likely used by the user for other sites as well. We will discuss in more detail in our final report why keyboard detection gives the adversary more of an ability to steal a user's password or other sensitive information.)

Once the technique is implemented, we will perform a web crawl applying the techique to some number of top Alexa websites. This should allow us to determine which of the most ``popular'' websites are tracking mouse movement and keyboard presses. Additionally, by collecting network traffic sent by the websites' client-side scripts, our data will indicate to whom the mouse and keyboard information is being sent.

After gathering the data from the web crawl, we will perform an analysis to try to find out how this mouse and keyboard information is being used. Our hypothesis is that this kind of tracking is being done quite widely, but mostly just for UX purposes. Web developers and UX designers have reason to learn exactly how users are using their site so that they may continuously rework the website design to make it easier to interact with. We would like to find out how many websites are using this information for illegitimate purposes. We believe we should gain some insight into the web developers' reasons for mouse and keyboard tracking by looking at the addresses to which the information is being sent. It is reasonable for the information to be sent back to the website's own server for usability testing purposes. Additionally, we are aware of some third-party mouse and keyboard tracking services that are commonly used by web developers for usability testing, such as ClickTale~\cite{clicktale}, Crazy Egg~\cite{crazyegg}, Inspectlet~\cite{inspectlet}, and Mouseflow~\cite{mouseflow}. %should those be endnotes instead of citations?

\section{This is Another Section}

Some embedded literal typset code might 
look like the following :

{\tt \small
\begin{verbatim}
int wrap_fact(ClientData clientData,
              Tcl_Interp *interp,
              int argc, char *argv[]) {
    int result;
    int arg0;
    if (argc != 2) {
        interp->result = "wrong # args";
        return TCL_ERROR;
    }
    arg0 = atoi(argv[1]);
    result = fact(arg0);
    sprintf(interp->result,"%d",result);
    return TCL_OK;
}
\end{verbatim}
}

Now we're going to cite somebody.  Watch for the cite tag.
Here it comes~\cite{BW,BW}.  The tilde character (\~{})
in the source means a non-breaking space.  This way, your reference will
always be attached to the word that preceded it, instead of going to the
next line.

\section{This Section has SubSections}
\subsection{First SubSection}

Here's a typical figure reference.  The figure is centered at the
top of the column.  It's scaled.  It's explicitly placed.  You'll
have to tweak the numbers to get what you want.\\

% you can also use the wonderful epsfig package...
\begin{figure}[t]
\begin{center}
\begin{picture}(300,150)(0,200)
\put(-15,-30){\special{psfile = fig1.ps hscale = 50 vscale = 50}}
\end{picture}\\
\end{center}
\caption{Wonderful Flowchart}
\end{figure}

This text came after the figure, so we'll casually refer to Figure 1
as we go on our merry way.

\subsection{New Subsection}

It can get tricky typesetting Tcl and C code in LaTeX because they share
a lot of mystical feelings about certain magic characters.  You
will have to do a lot of escaping to typeset curly braces and percent
signs, for example, like this:
``The {\tt \%module} directive
sets the name of the initialization function.  This is optional, but is
recommended if building a Tcl 7.5 module.
Everything inside the {\tt \%\{, \%\}}
block is copied directly into the output. allowing the inclusion of
header files and additional C code." \\

Sometimes you want to really call attention to a piece of text.  You
can center it in the column like this:
\begin{center}
{\tt \_1008e614\_Vector\_p}
\end{center}
and people will really notice it.\\

\noindent
The noindent at the start of this paragraph makes it clear that it's
a continuation of the preceding text, not a new para in its own right.


Now this is an ingenious way to get a forced space.
{\tt Real~$*$} and {\tt double~$*$} are equivalent. 

Now here is another way to call attention to a line of code, but instead
of centering it, we noindent and bold it.\\

\noindent
{\bf \tt size\_t : fread ptr size nobj stream } \\

And here we have made an indented para like a definition tag (dt)
in HTML.  You don't need a surrounding list macro pair.
\begin{itemize}
\item[]  {\tt fread} reads from {\tt stream} into the array {\tt ptr} at
most {\tt nobj} objects of size {\tt size}.   {\tt fread} returns
the number of objects read. 
\end{itemize}
This concludes the definitions tag.

\subsection{How to Build Your Paper}

You have to run {\tt latex} once to prepare your references for
munging.  Then run {\tt bibtex} to build your bibliography metadata.
Then run {\tt latex} twice to ensure all references have been resolved.
If your source file is called {\tt usenixTemplate.tex} and your {\tt
  bibtex} file is called {\tt usenixTemplate.bib}, here's what you do:
{\tt \small
\begin{verbatim}
latex usenixTemplate
bibtex usenixTemplate
latex usenixTemplate
latex usenixTemplate
\end{verbatim}
}


\subsection{Last SubSection}

Well, it's getting boring isn't it.  This is the last subsection
before we wrap it up.

\section{Acknowledgments}

A polite author always includes acknowledgments.  Thank everyone,
especially those who funded the work. 

\section{Availability}

It's great when this section says that MyWonderfulApp is free software, 
available via anonymous FTP from

\begin{center}
{\tt ftp.site.dom/pub/myname/Wonderful}\\
\end{center}

Also, it's even greater when you can write that information is also 
available on the Wonderful homepage at 

\begin{center}
{\tt http://www.site.dom/\~{}myname/SWIG}
\end{center}

Now we get serious and fill in those references.  Remember you will
have to run latex twice on the document in order to resolve those
cite tags you met earlier.  This is where they get resolved.
We've preserved some real ones in addition to the template-speak.
After the bibliography you are DONE.

{\footnotesize \bibliographystyle{acm}
\bibliography{bib}}


\theendnotes

\end{document}






